\documentclass{article}
\usepackage[utf8]{inputenc}
\usepackage{algorithmic}
\usepackage{listings}
\usepackage{amsmath}
\usepackage{xspace}
\usepackage{graphicx}
\usepackage{multirow}

\title{Hello \LaTeX World}

\author{Giovanni Salas\\
\small{Computer Engineering and Computer Science}\\
\small{Californai State University Long Beach}\\
\small{\texttt{giosalas25@gmail.com}}}
\date{August 25, 2020}

\begin{document}

\maketitle

\begin{abstract}
    This document is a model and  instructions for \LaTeX 'article' class.
\end{abstract}

\section{Introduction}
Welcome to the \LaTeX world.

\section{Ease of Use}

\subsection{Maintaining the Integrity of the Specifications}
The ‘article’ class is used to format your paper and style the text. All margins, column widths, line spaces, and text fonts are prescribed.

\section{Styling Guide}
\subsection{Abbreviations and Acronyms}
Define abbreviations and acronyms the first time they are used in the text, even after they have been defined in the abstract.

\subsection{Equations}
\begin{equation}
    \sum_{n=0}^{\infty}\frac{af^{n}}{n!}(x-a)^n
    \label{series}
\end{equation}
\eqref{series} is the famous Taylor series. Use “\eqref{series}”, not “Eq. \eqref{series}” or “equation \eqref{series}”, except at the beginning of a sentence: “Equation (1) is . . .”

Taylor series in a text would be $\sum_{n=0}^{\infty}\frac{af^{n}}{n!}(x-a)^n$

\subsection{Lists}
Bullet style list.
\begin{itemize}
    \item item 1
    \item item 2
    \item item 3
\end{itemize}
Number style list.
\begin{enumerate}
    \item item 1
    \item item 2
    \item item 3
\end{enumerate}

\subsection{Figures and Tables}
\paragraph{Positioning Figures and Tables} Figure captions should be below the figures; table heads should appear above the tables. Insert figures and tables after they are cited in the text. Use the abbreviation “Fig. 1”.

\begin{table}[h]
\caption{Table Type Styles}
\begin{center}
\begin{tabular}{|l|1|l|l|l|}
\hline
\multirow{2}{*}{\textbf{\begin{tabular}[c]{@{}l@{}}Table\\ Head\end{tabular}}} & \multirow{2}{*}{} & \multicolumn{3}{c|}{\textbf{Table Column Head}} \\ \cline{3-5} 
 &  & \textit{\textbf{Table column subhead}} & \textit{\textbf{Subhead}} & \textit{\textbf{Subhead}} \\ \hline
 &  &  &  &  \\ \hline
\end{tabular}
\end{center}
\end{table}

\begin{figure}[h]
\begin{center}
\includegraphics[scale=0.35]{csulb logo.png}
\caption{Working example}
\end{center}
\end{figure}

\subsection{Algorithms}
\begin{algorithmic}
    \STATE $i\gets 10$
    \IF{$i\ge 5$}
        \STATE $i\gets i-1$
    \ELSE
        \IF{$i\le 3$}
            \STATE $i\gets i+2$
        \ENDIF
    \ENDIF
\end{algorithmic}

\subsection{Source codes}
\lstinputlisting[language=Java, frame=tb]{Main.java}


\subsection{References}
Please number citations consecutively within brackets [1]. The sentence punctuation follows the bracket [2]. Refer simply to the reference number, as in [3] — do not use “Ref. [3]” or “reference [3]” except at the beginning of a sentence.

\begin{thebibliography}{}
\bibitem{eason} G. Eason, B. Noble, and I. N. Sneddon, “On certain integrals of LipschitzHankel type involving products of Bessel functions,” Phil. Trans. Roy. Soc. London, vol. A247, pp. 529–551, April 1955.
\bibitem{maxwell} J. Clerk Maxwell, A Treatise on Electricity and Magnetism, 3rd ed., vol. 2. Oxford: Clarendon, 1892, pp.68–73.
\bibitem{jacobs}  I. S. Jacobs and C. P. Bean, “Fine particles, thin films and exchange anisotropy,” in Magnetism, vol. III, G. T. Rado and H. Suhl, Eds. New York: Academic, 1963, pp. 271–350
\end{thebibliography}

\end{document}
